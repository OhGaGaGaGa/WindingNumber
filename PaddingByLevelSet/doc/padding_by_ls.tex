\documentclass[9pt]{extarticle}

\usepackage[hmargin=0.2in,vmargin=0.2in,columnsep=0.2in]{geometry}
\usepackage{times,parskip}
\usepackage{amsmath,amssymb}
\usepackage{cleveref}
\usepackage{tikz}
\usetikzlibrary{calc}
\newcommand{\tzma}[1]{\tikz[overlay,remember picture] \node (#1) {};}
\newcommand{\Cut}[1]{\mathcal{C}[#1]}
\newcommand{\taa}[2]{\draw[->,red,shorten >=3pt,shorten <=3pt] (#1.center) to (#2.center)}

\title{Padding by Level Set}
\begin{document}
\maketitle

\begin{abstract}
  In this note, we propose a reliable padding layer generation method
  with the proof about its robustness.
\end{abstract}

\section{Problem definition and decomposition}
The input is a tet mesh $\bar{M}$ (with 2D-manifold boundaries). Each
boundary face is associated with an integer to indicate the desired
layers below it.  We will generate a mesh composed of convex (tet,
prism, pyramid, hex) elements, and keep the input boundary untouched
to migrate kinds of labeling (e.g. boundary conditions) to the output
mesh.

The overall idea is to recursively cut the boundary elements by a
carefully desired level set.

\subsection{Initialization}

We first fit node-wise integer values $L$ to indicate the desired
number of layers about this node.  Roughly speaking, we assign $-1, 1$
to the nodes of boundary tet elements in $\bar{M}$, then use the level
set with values $0, 1/2, 2/3, \cdots, 1-1/k,\cdots$ to cut these
elements.

\subsection{Recursive}

Under the following mild assumption at $k$-th ($k=1,2,3,\cdots$) cut:
\begin{itemize}
\item the input hybrid but conforming mesh $M$ is only composed of
  tet, prism and pyramid with convex shape bounded by planar faces;
\item The hybrid mesh is tessellated from a tet mesh by piece-wise
  linear cuts.  In other words, each element $e \in M$ has its
  ancestor tetrahedron element $\mathcal{T}(e) \in \bar{M}$ in the
  original input tet mesh $\bar{M}$.
\item At least one node in each element is not on boundary.
\end{itemize}

\subsubsection{Piece-wise linear scalar field and integer level set}
The following operations are always feasible and lead to a new hybrid
mesh satisfying the above assumptions.

Giving the current integer values $L_i$ on the original tet mesh
$\bar{M}$ for its boundary nodes $n_i$:
\begin{itemize}
\item Assign $1$ to all nodes with value no less than
  $1$, and $\ell \triangleq 1-1/k$ to the others.
\item Assign $-1$ to nodes of the element which has descendant pyramid
  elements.
\item Assign $-1$ to all the inner nodes.
\item Linearly interpolate the piece-wise linear field $\psi$ on $\bar{M}$ to
  $M$.
\end{itemize}
This defines a $C^0$ piece-wise linear field $\phi$ on $M$, even in prism.

If an element is trivially $\ell$ in it, i.e. $\ell$-element, it means that
all its nodes are assigned $\ell$, i.e. all its node are on the boundary.
Contradict with assumption 3.

%% \item the boundary values are only $0$ and $k, k \in \mathbb{Z}^+$
%%   (``number of layers'', or even simply let $k=1$), i.e. uniform
%%   padding or not padding;
%% \end{itemize}
Then, we can prove
\begin{itemize}
  \item $\phi^{-1}(\ell)$ is piece-wise $2$-manifold (because of no
    $\ell$-element), planar if $\phi$ is piece-wise linear, in some
    input elements;
  \item All such elements, i.e. the padding region to be cut by
    $\phi^{-1}(\ell)$, is only composed of some boundary tets and
    prisms (no any pyramid) because the value in other elements
    must less than $\ell$.
  \item The result of cut is a conforming (due to $C^0$) mesh composed
    of tet, prism and pyramid.
    \begin{itemize}
      \item at least one node is inner node.
      \item pyramids are only generated at border of padding region,
        i.e. input boundary elements whose boundary values are not
        consistent.
      \item all are convex with planar faces.
      \item The new boundary elements are on the side of $\geq \ell$.
    \end{itemize}
\end{itemize}

\paragraph*{Non-uniform padding}
After cutting, we set $L_i \leftarrow L_i-1$ on the original input
tet.  If $\max(L_i)=0$, exit; otherwise, we go above.

\section{Algorithm}
In applications, sometimes we need a padding layer, i.e. thin cells around
boundary. Given a model $M$ (without loss of generality, we assume it has a
single connected boundary $\partial M$), to generate padding layers,
intuitively, with the help of signed distance $\phi_{\partial M}(x)$ (inner
negative, outer positive) to boundary $\partial M$, we can extract the
level set $S=\phi^{-1}(d)$ for small enough $d$ s.t. $S$ is connected.
Then, $P=\{x|\phi_{\partial M}(x)\le 0 \wedge  \phi_{S}(x)\ge 0\}$ is a
padding region. Discratize $P$ into cells, we would get desired the padding
layer. To generate multiple padding layers, we can take
$d_1<d_2<\cdots<d_n$ and obtain coresponding $S_i$. Due to the $C^0$ of
$\phi$, $S_i\ne S_j,i\ne j$. Therefore, $P_i=\{x|\phi_{S_{i-1}}(x)\le 0
  \wedge  \phi_{S_i}(x)\ge 0\}$ is the $i$-th padding region, where
$S_0=\partial M$.

In practice and on mesh, there are serveral requirements for padding
layers:
\begin{itemize}
  \item The padding should be partial. That is, only a subset of the
        boundary of the mesh should be padded.
  \item The discratized cells should be conforming.
  \item The discratized cells should be of restricted types.
  \item The number of layers should be controllable and be able to vary
        over the model.
\end{itemize}
Inspired by the above intuion, we can design a $C^0$ scalar function $f$, and take its level set to generate padding layers. The sclalar field should satisfy the following conditions:
\begin{itemize}
  \item Piecewise linear.
  \item The value of $f$ indicates the level of padding.
  \item Vertices have integer level of padding, i.e. $f(v)\in
          \mathbb{Z}$.
\end{itemize}

Then, we need to generate a new mesh $M'$, with an addition requirements:
\begin{itemize}
  \item For every position $x$ satisfying $f(x)\in \mathbb{Z}$, it should be on vertex, edge or face of $M'$, but not inside a cell. In other words, every cell $T$ can only cross at most $1$ level of padding layers:
\end{itemize}
\begin{equation}
  \max_{v\in T'} f(v)-\min_{v\in T'}\le 1
\end{equation}
This would allow users to control the number of padding layers and require us to cut the original mesh by level set $f^{-1}(K),K\in \mathbb{Z}$.

Then we would cut every cell crossing $[f,f+k]$
levels by level set $f^{-1}(f+k-1)$ into two cells crossing $[f,f+k-1]$ and
$[f+k-1,f+k]$ respectively. Recursively cutting on cells crossing more than $1$ levels. We would finally get the desired cells thus padding layers.

To simplify the problem, we assume that $|\{f(v) | v\in T \}|\le 2$ for
every initial cell $T$ and there are only tet cells in intial mesh. All the cutting situation is listed below:

{\small
\setlength{\jot}{20pt}
\begin{equation}
  \begin{aligned}
    T(f,f+k+1,\tzma{T13}f+k+1,f+k+1)               & \rightarrow T(f,f+k,\tzma{oT13}f+k,f+k),\boxed{W(f+k,f+k,f+k,f+k+1,f+k+1,f+k+1)}                        \\
    T(f,f,\tzma{T22}f+k+1,f+k+1)                   & \rightarrow W(f,f+k,f+k,\tzma{oW1212}f,f+k,f+k),\boxed{W(f+k,f+k,f+k+1,f+k,f+k,f+k+1)}                  \\
    T(f,f,\tzma{T31}f,f+k+1)                       & \rightarrow W(f,f,f,\tzma{oW33}f+k,f+k,f+k),\boxed{T(f+k,f+k,f+k,f+k+1)}                                \\
    T(f,f+k_1, \tzma{T112}f+k_1+k_2,f+k_1+k_2)     & \rightarrow T(f,f+k_1,\tzma{o1T13}f+k_1,f+k_1),P(f+k_1,f+k_1,f+k_1,\tzma{oP32}f+k_1+k_2,f+k_1+k_2)      \\
    T(f,f+k_1,\tzma{T121}f+k_1,f+k_1+k_2)          & \rightarrow T(f,f+k_1,\tzma{o2T13}f+k_1,f+k_1), T(f+k_1,f+k_1,\tzma{oT211}f+k_1,f+k_1+k_2)              \\
    T(f,f,\tzma{T211}f+k_1,f+k_1+k_2)              & \rightarrow P(f,f,\tzma{oP23}f+k_1,f+k_1,f+k_1), T(f+k_1,f+k_1,\tzma{oT31}f+k_1,f+k_1+k_2)              \\
    T(f,f+k_1,\tzma{T1111}f+k_1+k_2,f+k_1+k_2+k_3) & \rightarrow T(f,f+k_1,\tzma{o3T13}f+k_1,f+k_1),P(f+k_1,f+k_1,f+k_1,\tzma{oP311}f+k_1+k_2,f+k_1+k_2+k_3) \\
    W(f,f,f,\tzma{W33}f+k+1,f+k+1,f+k+1)           & \rightarrow W(f,f,f,\tzma{o1W33}f+k,f+k,f+k),\boxed{W(f+k,f+k,f+k,f+k+1,f+k+1,f+k+1)}                   \\
    W(f,f+k+1,f+k+1,\tzma{W1212}f,f+k+1,f+k+1)     & \rightarrow W(f,f+k,f+k,\tzma{o1W1212}f,f+k,f+k),\boxed{H(f+k,f+k,f+k,f+k,f+k+1,f+k+1,f+k+1,f+k+1)}     \\
    W(f,f,f+k+1,\tzma{W2121}f,f,f+k+1)             & \rightarrow W(f+1,f+1,f+k+1,\tzma{o1W2121}f+1,f+1,f+k+1),\boxed{H(f,f,f+1,f+1,f,f,f+1,f+1)}             \\
    P(f,f,f,\tzma{P311}f+k_1,f+k_1+k_2)            & \rightarrow ?(f,f,f,f+k_1,f+k_1,f+k_1), T(f+k_1,f+k_1,\tzma{o1T31}f+k_1,f+k_1+k_2)                      \\
    P(f-1,f-1,\tzma{P32}f-1,f+k,f+k)               & \rightarrow W(f,f,f+k,\tzma{oW2121}f,f,f+k),\boxed{?W(f-1,f-1,f-1,f,f,f,f)}                             \\
    P(f, f, \tzma{P23}f+k,f+k,f+k)                 & \rightarrow W(f,f+k,f+k\tzma{o2W1212},f,f+k,f+k),\boxed{?W(f+k+1,f+k+1,f+k+1,f+k,f+k,f+k,f+k)}
  \end{aligned}
  \begin{tikzpicture}[overlay,remember picture,out=270,in=270,distance=1cm]
    \draw[->,red,shorten >=3pt,shorten <=3pt] (oT13.center) to (T13.center);
    \draw[->,blue,shorten >=3pt,shorten <=3pt] (oW1212.center) to (W1212.center);
    \draw[->,blue,shorten >=3pt,shorten <=3pt] (oW33.center) to (W33.center);
    \draw[->,red,shorten >=3pt,shorten <=3pt] (o1T13.center) to (T13.center);
    \draw[->,orange,shorten >=3pt,shorten <=3pt] (oP32.center) to (P32.center);
    \draw[->,red,shorten >=3pt,shorten <=3pt] (o2T13.center) to (T13.center);
    \draw[->,red,shorten >=3pt,shorten <=3pt] (oT211.center) to (T211.center);
    \draw[->,orange,shorten >=3pt,shorten <=3pt] (oP23.center) to (P23.center);
    % \draw[->,red,shorten >=3pt,shorten <=3pt] (oT121.center) to (T121.center);
    \draw[->,red,shorten >=3pt,shorten <=3pt] (o3T13.center) to (T13.center);
    \draw[->,orange,shorten >=3pt,shorten <=3pt] (oP311.center) to (P311.center);
    \draw[->,blue,shorten >=3pt,shorten <=3pt] (o1W33.center) to (W33.center);
    \draw[->,blue,shorten >=3pt,shorten <=3pt] (o1W1212.center) to (W1212.center);
    % \draw[->,orange,shorten >=3pt,shorten <=3pt] (o1P32.center) to (P32.center);
    \draw[->,red,shorten >=3pt,shorten <=3pt] (o1T31.center) to (T31.center);
    \draw[->,blue,shorten >=3pt,shorten <=3pt] (oW2121.center) to (W2121.center);
    \draw[->,blue,shorten >=3pt,shorten <=3pt] (o1W2121.center) to (W2121.center);
    \draw[->,blue,shorten >=3pt,shorten <=3pt] (oW1212.center) to (W1212.center);
    \draw[->,red,shorten >=3pt,shorten <=3pt] (o3T13.center) to (T13.center);
    \draw[->,blue,shorten >=3pt,shorten <=3pt] (o2W1212.center) to (W1212.center);
    
  \end{tikzpicture}
\end{equation}
},
where $k\in\mathbb{Z}^{+}$, \boxed{boxed} is terminated cell.

It is easy to see the recursion is teminatable and cell types are
restricted to Tet, Wedge and Hex.


---old text---

\section{Planar cut a tet}
Each node $n_i$ of a tet is labeled by $s_i\in\{-1,0,1\}$ for inside,
boundary or outside.  Then, a case can be identified as a string $c\in
  \{-1,0,1\}^4$, and $c$ is symmetric to $-c$ up to a flipping of inside
and outside.  To remove more symmetric cases, we assume $s_1\leq
  s_2\leq s_3\leq s_4$.  We denote the type of sub-element generated by
a cut as $4,5,6$ for tetrahedron, pyramid, triangular prism (i.e. by
the number of nodes).  Result of cut are also two string encodes the type
of sub-elements in the two sides.

All cases are listed in~\Cref{tab:cases}

\begin{table}[htb]
  \centering
  \begin{tabular}{|c|ccc|}
    \hline
    $s$                                 & -1 side & 1side & note \\
    \hline
    $\forall s_i \leq 0$ (i.e. -1-1-10) & 4       &       &      \\
    \hline
    -1-1-1 1                            & 6       & 4     &      \\
    -1 1 1 1                            & 4       & 6     &      \\
    \hline
    -1-1 0 1                            & 5       & 4     &      \\
    -1 0 1 1                            & 4       & 5     &      \\
    \hline
    -1-1 1 1                            & 6       & 6     &      \\
    \hline
    $\forall s_i \geq 0$ (i.e. 0 0 1 1) &         & 4     &      \\
    \hline
  \end{tabular}
  \caption{Cases}
  \label{tab:cases}
\end{table}

\section{Level set}

As long as the field is piece-wise linear, $C^0$ continuous with
non-degenerate gradient, the level set is conforming and piece-wise
planar.  It can also be proved that all the elements are convex and
have positive volume.

\section{Proof}
\label{sec:proof}

Under the following assumptions:
\begin{itemize}
  \item Non-degenerate tetrahedral mesh with 2D-manifold boundaries.
  \item The scalar values on the four nodes of a tetrahedron are not
        same (non-degenerate gradient).
\end{itemize}

We have the following claim:
\begin{itemize}
  \item The regions on both side of the level set are only composed of
        tetrahedron, triangular prism and pyramid.
  \item All such elements are convex and bounded by planar faces with
        positive volume.
\end{itemize}

If we further restrict
\begin{itemize}
  \item All inner nodes of the tetrahedral mesh has the negative value.
  \item All boundary nodes of the tetrahedral mesh has non-negative
        value.
\end{itemize}
It can be proved:
\begin{itemize}
  \item The level set only passes the first layer of the tetrahedral mesh.
\end{itemize}

\section{Conclusion}

Under the assumption and restriction in~\Cref{sec:proof}, it is
impossible to fail at generating \textbf{a conforming hybrid mesh
  composed of tetrahedral, triangular prisms and pyramids with convex,
  planar boundary and positive volume}.

\end{document}
